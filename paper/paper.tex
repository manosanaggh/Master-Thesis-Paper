%%%%%%%%%%%%%%%%%%%%%%%%%%%%%%%%%%%%%%%%%%%%%%%%%%%%%%%%%%%%%%%%%%%%%%%%%%%%%%%%
% Template for USENIX papers.
%
% History:
%
% - TEMPLATE for Usenix papers, specifically to meet requirements of
%   USENIX '05. originally a template for producing IEEE-format
%   articles using LaTeX. written by Matthew Ward, CS Department,
%   Worcester Polytechnic Institute. adapted by David Beazley for his
%   excellent SWIG paper in Proceedings, Tcl 96. turned into a
%   smartass generic template by De Clarke, with thanks to both the
%   above pioneers. Use at your own risk. Complaints to /dev/null.
%   Make it two column with no page numbering, default is 10 point.
%
% - Munged by Fred Douglis <douglis@research.att.com> 10/97 to
%   separate the .sty file from the LaTeX source template, so that
%   people can more easily include the .sty file into an existing
%   document. Also changed to more closely follow the style guidelines
%   as represented by the Word sample file.
%
% - Note that since 2010, USENIX does not require endnotes. If you
%   want foot of page notes, don't include the endnotes package in the
%   usepackage command, below.
% - This version uses the latex2e styles, not the very ancient 2.09
%   stuff.
%
% - Updated July 2018: Text block size changed from 6.5" to 7"
%
% - Updated Dec 2018 for ATC'19:
%
%   * Revised text to pass HotCRP's auto-formatting check, with
%     hotcrp.settings.submission_form.body_font_size=10pt, and
%     hotcrp.settings.submission_form.line_height=12pt
%
%   * Switched from \endnote-s to \footnote-s to match Usenix's policy.
%
%   * \section* => \begin{abstract} ... \end{abstract}
%
%   * Make template self-contained in terms of bibtex entires, to allow
%     this file to be compiled. (And changing refs style to 'plain'.)
%
%   * Make template self-contained in terms of figures, to
%     allow this file to be compiled. 
%
%   * Added packages for hyperref, embedding fonts, and improving
%     appearance.
%   
%   * Removed outdated text.
%
%%%%%%%%%%%%%%%%%%%%%%%%%%%%%%%%%%%%%%%%%%%%%%%%%%%%%%%%%%%%%%%%%%%%%%%%%%%%%%%%

\documentclass[letterpaper,twocolumn,10pt]{article}
\usepackage{usenix2019_v3}

% to be able to draw some self-contained figs
\usepackage{tikz}
\usepackage{amsmath}
\usepackage{graphicx}
\usepackage{filecontents}
\usepackage{xspace}
\usepackage{color, soul, colortbl}
\usepackage{multirow}
\usepackage{hyperref}
\urlstyle{same}
\usepackage{subfigure}
\usepackage{subcaption}
\usepackage{tabularx,booktabs}
\usepackage{url}

\newcommand{\note}[1]{{\color{red}{\textbf{#1}}}}
\newcommand{\comment}[1]{{}}
\newcommand*\circled[1]{\tikz[baseline=(char.base)]{
            \node[shape=circle,fill,inner sep=0.4pt] (char) {\textcolor{white}{#1}};}}

\def\tick{\tikz\fill[scale=0.4](0,.35) -- (.25,0) -- (1,.7) -- (.25,.15) -- cycle;} 
\newcommand\ChangeRT[1]{\noalign{\hrule height #1}}

%-------------------------------------------------------------------------------
\begin{document}
%-------------------------------------------------------------------------------

%don't want date printed
\date{}

% make title bold and 14 pt font (Latex default is non-bold, 16 pt)
\title{\Large \bf Analysis of Server Throughput For Managed Big Data Analytics
    Frameworks.}
}

%Overcoming the \note{M}emory \note{B}ound of Big
%Data Analytics to \note{I}mprove \note{S}erver \note{T}hroughput
%\note{U}sing \note{F}ast \note{S}torage Devices 

%for single author (just remove % characters)
\author{
{\rm Emmanouil Anagnostakis}\\
Institute of Computer Science (ICS), Foundation of Research and Technology -- Hellas (FORTH), Greece \\
Computer Science Department, University of Crete, Greece
}
%\and
%{\rm Second Name}\\
%Second Institution
% copy the following lines to add more authors
% \and
% {\rm Name}\\
%Name Institution
%} % end author

\maketitle
\begin{abstract}

    Managed big data analytic frameworks require a lot of memory to process
    large amounts of data.
    %
    The memory pressure that arises during the processing of
    large amounts of data
    prevents the frameworks from utilizing the CPU thus leading to
    low server throughput.
    %
    This pressure leads to long Garbage Collection
    (GC) cycles
    leaving no room for useful work. 
    In this paper, we conduct an analysis of server
    throughput for managed big data analytics using smart heap
    offloading to fast storage devices in order to reduce memory pressure.
    %
    We use a framework called TeraHeap that moves objects from the Java
    managed heap to a secondary heap over a fast storage device
    thereby freeing
    up heap memory and reducing memory pressure without suffering from
    storage latencies.  
    %
    By overcoming the memory bound we leave space
    to the applications for more CPU utilization. 
    %
    We present a
    detailed methodology for running Apache Spark and Giraph using TeraHeap, which significantly improves
    server throughput for managed big data analytics. The methodology includes
    conducting a research about the needs of Big data analytics when using offheap mechanisms
    such as the need for Java Heap or for IO Cache, breaking down the execution time and explaining
    every different metric, provinding results such as server throughput...\note{jk: Why
    your methodology is important. What aspects are you investigate.
    You should write here 2-3 sentences trying to show that you put
    effort for the methodology}. TeraHeap is implemented in Oracle's OpenJDK8. In this paper we
    evaluate its performance against Native Spark and Giraph using various
    workloads of the Spark Bench suite and graphalytics library on a real-world cluster.
    Our primary focus is to examine its performance under the colocation of
    multiple instances. 
    Our experimental results show that reducing memory pressure
    by offloading the heap to fast storage devices
    significantly improves server throughput while reducing
    memory usage against native Spark and Giraph. We also include
    results to show that TeraHeap can reduce monetary cost if
    deployed in a world cluster like Amazon's EC2 or Google Cloud
    Platform or Microsoft Azure Cloud which are available to everyone. 
    \note{jk: Overall comments for the abstract. Abstract should be
    maximum 250 - 300 words. Split it in two or three paragraphs.
    First paragraph state the problem. This paragraph should be
    maximum 3-4 sentences. Next paragraph you should explain what this
    paper provides. In our case is the methodology and evaluation.
    Finally, in the third paragraph you should mention the most
    important results. What the author should rembember.}
\end{abstract}

\section{Introduction}
\label{sec:intro}
With the exponential growth of data in various fields such as
healthcare and social media, managed big data frameworks (e.g, Apache
Spark \cite{Spark} and Apache Giraph \cite{Giraph}) require large
amount of DRAM per core for data processing. During the processing, they generate large
amount of objects in the managed heap that span multiple computation
stages. The memory pressure that arises in the managed heap leads to
frequent garbage collection (GC) cycles. Frequent GCs waste CPU cycles 
and prevent application execution.

On the one hand, to reduce the frequency of GC and optimize performance, big data
frameworks offload objects from the managed heap to storage devices. However, these
objects need to be serialized to byte streams to be stored in the storage
device or to be deserialized into memory objects to be loaded back to memory. 
This practice leads to high serialization/deserialization overhead.
On the other hand infrastructure providers, trying to address the same problem increase memory per framework instance that runs in the server. This leaves CPU cores underutilized. 

Co-locating workloads aims to increase available resource utilization
thus increasing the throughput in server. 
In order to maximize throughput, the number of instances
increase to utilize all available DRAM. The result of this
practice is that the underlying machine runs out of memory, while the
 overhead of GC and S/D is still high. The remaining GC and S/D
overheads lead to the problem of wasting the CPU resources
to do unuseful work. This leads to the conclusion that the avalaible memory per core is
not enough for the Garbage Collector, S/D and the application.

The memory per core problem can better be understood when looking at
the resource usage and the characteristics of the servers of big
companies e.g. Alibaba and Facebook. When looking at the results of
Alibaba's traces analyses (\cite{Alibaba}, \cite{Alibaba1},
\cite{Alibabacolocated}) we see that memory usage is at an average of
80\%, while CPU usage stays at 40\%. This trace clearly shows that
DRAM utilization is high, while the CPU is under-utilized. In
Facebook's Twine presentation \cite{Twine}, they used a cluster of
machines where each machine had 40 cores and 80 GB DRAM. This means
that ratio of GB for memory per core was 2. The same ratio is shown in Facebook's Yosemite
\cite{Yosemite}. This shows that memory
capacity for each core is low while DRAM usage is high compared to the CPU usage. Most of the time many the CPU cores are
going to be idle because a few of them will be enough to carry out the
work.

To address the problem of DRAM capacity limitation, recent work
proposed solutions that extend the managed heaps over local flash
storage devices (e.g., NVMe SSD) or remote memory. On the one hand,
TMO \cite{TMO} offloads cold memory to fast storage devices using
a memory scheduling mechanism. On the other hand, CFM \cite{CFM}
utilizes remote DRAM as swap memory in order to increase total memory capacity
and reduce memory pressure. Of both works, only CFM shows
evaluation against managed big data analytics frameworks. However, this evaluation
includes only one Spark workload and is not focused on analytics.

This thesis provides a methodological analysis of server throughput 
focused on managed big data analytics frameworks.
We investigate the off-heap direction of offloading the objects from the
managed heap to fast storage devices.
Specifically, we use TeraHeap (TH) \cite{TeraHeap}, a secondary managed
memory-mapped heap over an NVMe storage device, which is used to hold
the long lived objects instead of the main managed Java Heap. TeraHeap
1) eliminates Serialization/Deserialization overheads posed by this
kind of frameworks when moving data off-heap to/from fast storage
devices 2) reduces GC pauses drastically over the secondary heap. By
using TeraHeap, we aim to investigate the impact of reducing GC and S/D
to server throughput under workload co-location compared to Native Spark
and Giraph. We divide all the available DRAM
in our machine to 2,4 and 8 even budgets to run the co-located instances.
First we run each instance isolated to analyze performance and be able to study the interference when adding more
co-located instances. We run each individual workload with a different Spark or Giraph instance in a cgroup.
We do this to limit the memory budget for each instance. Memory budget is
the summary of Java Heap, IO Cache (Linux Page Cache) and JVM native memory. We choose
the Java Heap (H1) ratio over the total DRAM budget based on RedHat's decisions
for running containers as a baseline. We also run experiment with more Page Cache (PC) ratio than H1
to investigate Page Cache affection to the performance. We show performance of both Native Spark-Giraph and Spark-Giraph with TH in 3 different
memory per core scenarios, 4 GB per core, which is the current trend and 8 and 16 GB per core 
as possible future trends. We evaluate both offloading techniques by running 2 widely used
managed big data frameworks, Apache Spark and Giraph. We
specificaly run 4 different workloads with Spark with 4 and 8 GB per core.
We run 2 different workloads with Giraph with 8 and 16 GB per core.
We compare TeraHeap with the native Spark and Giraph distributions under workload
co-location and analyze their performance using several metrics like
GC, S/D, I/O and CPU utilization. Finally, we estimate the cost of running these
experiments in public world clusters like Amazon EC2, Google Cloud Platform (GCP)
and Microsoft Azure Cloud to see possible benefits of either of the two techniques.

Our experimental results show that there is indeed a problem with memory per core ratio
for managed big data frameworks.
A solution is to move the managed heap over fast storage devices to avoid both GC and S/D like TeraHeap and Panthera \cite{Panthera}.
Our experimental results show that reducing GC and S/D by offloading the heap to fast storage devices improves effective CPU utlization up to 59\% in CPU cycles for Spark and also leaves place to run more co-located instances in the server for both frameworks.
Finally, we also include a cost estimation to show that reducing GC and S/D could reduce monetary cost by up to 50\% for running big data
analytics, in a world cluster like EC2, GCP or Microsoft Azure Cloud, which are available to
the public.

To summarize, this thesis makes the following contributions: 
\begin{itemize}
    \item{A detailed methodology for running co-located Apache Spark and Giraph
        workloads with or without TeraHeap. 
	We show how to pick the DRAM budget for each co-located instance and how to divide this budget
		for the different memory needs of a group of processes.
		}

    \item{A comprehensive evaluation using the above methodology.
	    We run 4 Spark and 2 Giraph workloads in 3 memory per core scenarios
	    analyzing different aspects of performance like GC, S/D, I/O, average throughput, CPU utilization and CPU cycles. We show the interference impact of running multiple co-located managed big data frameworks workloads and show that reducing GC and S/D increases application throughput and utilizes the CPU in a better and more efficient way for Spark. We also show that decreasing GC and S/D, increases the number of co-located instances that can be executed in the server.}

    \item{A cost estimation of running our experiments in real-world
        cloud platforms like Amazon EC2, Google Cloud Platform and Microsoft
        Azure.}
\end{itemize}

%
\end{itemize}

\section{Related Work}

\note{jk: Try to categorize the related work and add citations using
"cite\{\} command". Try to use the name of the system in each paper
and not authors name. When you discuss the work in each category try
to show paper in chronological order.}

Several studies have been conducted to improve the performance of big
data processing systems. One approach is to utilize memory-aware task
co-location to improve Spark application throughput, which has been
investigated by Marco et al. in [3]. \note{jk: Provide more detail
about their technique} 
Meanwhile, in [4], Kirisame et
al. proposed optimal heap limits to reduce browser memory use.
\note{jk: Again add more details.} Another
research direction is to leverage far memory to improve job
throughput, as studied by Amaro et al. in [5]. \note{jk: How this work
is different for the Marco et al. work.}
To facilitate memory
offloading in datacenters, Weiner et al. presented TMO, a transparent
memory offloading system in [6] \note{jk: what they do. Give more
details. now is like shopping list}. In cloud computing platforms, Sharma
et al. proposed per-VM page cache partitioning to improve performance
in [7]. Chen and Wang introduced Spark on Entropy, a reliable and
efficient scheduler for low-latency parallel jobs in heterogeneous
clouds, in [8]. Thamsen et al. developed Mary, Hugo, and Hugo*, three
learning-based schedulers for distributed data-parallel processing
jobs on shared clusters in [9] \note{jk: is this related of what we
are doing? Maybe yes but you need to provide more context}. Additionally, Bhimani et al. proposed
a lightweight virtualization framework for accelerating big data
applications on enterprise cloud in [10], while Zhang et al. focused
on understanding and improving disk-based intermediate data caching in
Spark in [11]. Finally, Intasorn et al. investigated using compression
tables to improve HiveQL performance with Spark in a case study on
NVMe storage devices in [12]. 
 
\note{I thin this paragraph does not help. It sounds like the previous
study make a big effort and provide analysis. So, I guess, why do I
need to read this paper? What is the new that I will learn. Can we
show the open research questions/problems that the previous studies do
not target?}
These studies demonstrate a variety of approaches for optimizing big
data processing systems, ranging from memory-aware task co-location
and memory offloading to scheduler design and virtualization
frameworks. The findings from these studies can provide insights and
guidance for future research in the field of big data processing.

\section{Experimental Methodology}
\label{sec:method}

In our methodology we answer the following questions:
\begin{itemize}
	%\item{What are our server characteristics?}
	%\item{What are our configurations of Spark and Giraph with and without using TeraHeap?}
	\item{What workloads did we choose to run for our experiments and why?}
	\item{Are Spark and Giraph in need of more Java Heap or more cache for IO?}
	%\item{What kind of metrics should someone use to be accurate when measuring performance?}
	%\item{Why would someone choose to run the workloads concurently and not one by one?}
	\item{Is cost a contributing factor to pursuing higher throughput for a server?}
\end{itemize}

\end{itemize}
\begin{table}[t!]
  \centering
  \caption{Server Specifications}
  \label{tab:server-specs}
  \begin{tabular}{|c|c|}
    \hline
    \textbf{Component} & \textbf{Specification} \\
    \hline
    Memory & 8x DDR4 32-GB \\
    CPU & 2x Intel Xeon E5-2630 2.4 GHz \\
    Cores per CPU & 16 \\
    NUMA Islands & 2 \\
    Cores per NUMA island & 16 \\
    L1 Cache & 512 KB \\
    L2 Cache & 2 MB \\
    L3 (LLC) Cache & 20 MB \\
    Storage & 2x 1.8 TB KVS NVMe \\
    \hline
  \end{tabular}
\end{table}

\subsection{Server Characteristics}
In \ref{tab:server-specs} we see the characteristics of our server.
The server used in our experiments is a high-performance machine with
hardware specifications found in real-world clusters.
It is equipped with 8x DDR4 32-GB 2.4 GHz 64-bit DIMMs, providing a
total of 256 GB of memory. The DDR4 memory technology is known for its
high bandwidth and low power consumption, making it ideal for
data-intensive applications like big data analytics. The server also
features 2x Intel Xeon E5-2630 2.4 GHz 16-core 64-bit CPUs, split into 2
NUMA islands. Each core has 512 KB L1, 2 MB L2, and 20 MB L3 (LLC) cache.
The Xeon E5-2630 CPU is a high-performance processor designed for data
centers, offering a high core count, high clock speed, and advanced
features like hyper-threading and Turbo Boost. The large L3 cache
helps reduce memory latency, enabling faster data access for CPU-bound
workloads. In addition to the powerful CPUs and memory, the server
also has 2x KVS NVMe storage devices. NVMe is a high-performance
storage technology that uses PCIe to connect directly to the CPU,
providing low latency and high throughput. The KVS (Key-Value Store)
storage devices are designed for fast, random access to data, making
them ideal for storing and retrieving large amounts of data in big
data applications. Overall, the server's hardware specifications make
it a powerful platform for conducting experiments on managed big data
analytics and evaluating the performance of TeraHeap. 

\subsection{Native Spark Configuration}
We use Spark v3.3.0 (\cite{Building}, \cite{Tuning}, \cite{Conf}, \cite{Monitoring}) with Kryo Serializer \cite{Kryo}, a state-of-the-art highly
optimized S/D Library for Java that Spark recommends. We run Spark
with Native OpenJDK8 \cite{JDK8} as a baseline. We use the Parallel Scavenge
garbage collector which is the one TeraHeap is implemented for.
Parallel Scavenge is also the go-to collector for applications that
need high throughput like Spark. We use an executor with 8 executor
threads for each instance of Spark we deploy on our server. For
Parallel Scavenge, we use 8 GC Threads for minor GC and the
single-threaded old generation collector. Spark storage level
is configured to MEMORY-AND-DISK to place executor memory (heap) in DRAM and cache RDDs \cite{RDD}
in the on-heap cache, up to 50\% of the total heap size. Any remaining
RDDs are serialized in the off-heap cache over an NVMe SSD. This
device is also used by Spark for shuffling. We run each instance of
Spark in a cgroup containing two JVM instances, one for Spark driver
and one for Spark executor and all the processes needed to measure
performance for this instance. Each cgroup has a limited DRAM Budget.
A part of the budget is the capacity of the Java Heap which, for the
rest of the paper, we call H1. We do this to be sure that
every instance of Spark running on our server has a fair amount of
DRAM available for it to use. We choose to try two different amounts
for H1, 40\% and 80\% of total DRAM budget. 80\% heap/DRAM is the go-to
percentage for RedHat in its datacenters [?]. What
remains is used by JVM for Native memory (i.e. CodeCache), Spark and Giraph
drivers and for the operating system's Page Cache (PC) when using TeraHeap. 
In order to avoid inter-NUMA island interference, 
we shut down the 16 cores belonging to the second island
thus leaving 16 active cores. We also turn off the swapper, because it
adds significant overhead and makes it difficult to understand the
results of the experiments conducted.

\subsection{Native Giraph Configuration}
Empty

\subsection{TeraHeap background and Spark-Giraph configurations for TeraHeap}
\subsubsection{What is TeraHeap?}
TeraHeap is a high-capacity managed heap that is memory-mapped over a
fast storage device (preferrably block-addresable NVMe or
byte-addresable NVM). The high speeds these kind of devices operate
in, erase any overhead caused by the use of MMIO. TeraHeap is designed
as an extension of the main Java Heap. It holds specific long-lived
objects that have the same lifetime span. This enables TeraHeap to
operate as a GC-free heap that can delete entire regions of objects at
once without the need to scan the heap over and over again for dead
objects. This would be a performance kill as it would require scans
over the storage device. The two main contributions of TeraHeap are
the following: 1) MMIO keeps the objects that reside in the storage
device deserialized, thus eliminating the need for
Serialization/Deserialization 2) As discussed, TeraHeap reduces GC overheads
without wasting DRAM by avoiding scans on long-lived objects.

\subsubsection{Spark Configuration}
The configuration for TeraHeap is pretty much the same as for Native
Spark, with some necessary differences. TeraHeap
is mapped to a different storage device (NVMe) than that Spark is
using for shuffling. We do this in order for TeraHeap to utilize its
device to its fullest. MMIO allows TeraHeap Spark to run in
MEMORY-ONLY storage level as Spark remains unaware of using any device and
the OS takes control of the I/O. 

\subsubsection{Giraph Configuration}
For Giraph we map TeraHeap to a different NVMe storage device that the one we
use for Zookeeper. TeraHeap works in the same way as Spark in MEMORY-ONLY storage level
as Giraph is unaware of the presence of a second heap.

\subsection{Workloads}
For our experiments, we selected four specific workloads from two
different categories of the Spark Bench suite \cite{Spark-Bench}: Page Rank (PR) and Connected
Component (CC) from GraphX and Linear Regression (LinR) and Logistic Regression (LogR)
from MLLib. For Giraph we choose PageRank, Weakly Connected Component and Community detection 
using label propagation (CDLP) from LDBC Graphalytics \cite{ldbc}. The primary reason for selecting these workloads for Spark is that
they represent different types of algorithms: PR, CC 
and are graph-based workloads, while LinR and LogR are machine learning
workloads. Giraph is a graph processing framework so we used only graph workloads. Furthermore, all of them are
well-established workloads that are commonly used for benchmarking big
data analytics systems, making them a suitable choice for our
experiments. Overall, the selection of these workloads allows us to
evaluate the performance of TeraHeap in a variety of contexts and
provide insights into the effectiveness of such an approach for improving
server throughput in managed big data analytics systems.

\subsubsection{PageRank}
PageRank is a widely used graph-based algorithm that measures the
importance of nodes in a network. It has become a popular benchmark
for evaluating the performance of distributed systems, including big
data analytics systems like Apache Spark. PageRank is computationally
intensive and requires significant memory and I/O resources, making it
a suitable workload for evaluating the performance of our TeraHeap
for improving server throughput. Additionally, PageRank is a
common algorithm in real-world applications, such as search engines
and social networks, making it relevant for practical use cases.

\subsubsection{LinearRegression}
LinearRegression is a machine learning algorithm that is used to
predict numerical values based on input data. It is a well-known and
widely used algorithm in machine learning, and is commonly used for
regression analysis in fields such as economics, finance, and
engineering. LinearRegression is computationally intensive and
requires significant memory and I/O resources, making it a suitable
workload for evaluating the performance of TeraHeap for
improving server throughput. Furthermore, the inclusion of a machine
learning workload like LinearRegression allows us to investigate the
performance of TeraHeap across different types of big data
analytics tasks and gain insights into the effectiveness of TeraHeap
for improving server throughput in a range of contexts.

\subsubsection{Logistic Regression}
LogisticRegression is a machine learning algorithm that is used to
model the probability of a binary or categorical outcome based on one
or more independent variables. It is commonly used in predictive
analytics to classify data based on historical data. In Spark-bench,
LogisticRegression is implemented as a machine learning workload,
where the dataset is represented as an RDD of feature vectors and
labels. The LogisticRegression workload involves training a logistic
regression model on the dataset, using an iterative optimization
algorithm such as gradient descent. The workload is computationally
intensive and requires a significant amount of memory to store the
dataset and model parameters.

\subsubsection{Connected Component}
ConnectedComponent is a graph algorithm that is used to identify the
connected components of a graph. It is commonly used in social network
analysis to identify clusters of users with similar interests or
relationships. In Spark-bench, ConnectedComponent is implemented as a
graph processing workload, where the graph is represented as an RDD of
edges and vertices. The ConnectedComponent workload involves iterating
over the graph, identifying the connected components of each node, and
merging the components as necessary. The workload is computationally
intensive and requires a significant amount of memory to store the
graph.

\subsubsection{Weakly Connected Component}
empty

\subsubsection{Community Detection Label Propagation}
empty

\subsubsection{}
\begin{figure}[thbp]
	\centering
    \includegraphics[width=\linewidth]{./fig/gcs_linr_h1_native.png}
    \caption{Number of GCs over time for H1 Linear Regression Native
    Spark investigation.}
    \label{fig:gcs_linr_h1_native}

    \includegraphics[width=\linewidth]{./fig/gcs_linr_h1_th.png}
    \caption{Number of GCs over time for H1 Linear Regression TeraHeap
    Spark investigation.}
    \label{fig:gcs_linr_h1_th}
\end{figure}

\begin{figure}[thbp]
    \includegraphics[width=\linewidth]{./fig/gcs_linr_pc_th.png}
    \caption{Number of GCs over time for Page Cache Linear Regression
    TeraHeap Spark investigation.}
    \label{fig:gcs_linr_pc_th}  

    \includegraphics[width=\linewidth]{./fig/gcs_pr_h1_th.png}
    \caption{Number of GCs over time for H1 Page Rank TeraHeap Spark
    investigation.}
    \label{fig:gcs_pr_h1_th}
\end{figure}

\begin{figure}[thbp]
    \includegraphics[width=\linewidth]{./fig/gcs_pr_h1_native.png}
    \caption{Execution time breakdown for H1 Linear Regression Native
    Spark investigation.}
    \label{fig:gcs_pr_h1_native}

    \includegraphics[width=\linewidth]{./fig/gcs_pr_pc_th.png}
    \caption{Execution time breakdown for PageCache Page Rank TeraHeap
    Spark investigation.}
    \label{fig:gcs_pr_pc_th}
\end{figure}

\begin{figure}[thbp]
    \includegraphics[width=\linewidth]{./fig/linr_h1_native.png}
    \caption{Execution time breakdown for H1 Linear Regression Native
    Spark investigation.}
    \label{fig:linr_h1_native}

    \includegraphics[width=\linewidth]{./fig/linr_h1_th.png}
    \caption{Execution time breakdown for H1 Linear Regression
    TeraHeap Spark investigation.}
    \label{fig:linr_h1_th}
\end{figure}

\begin{figure}[thbp]
    \includegraphics[width=\linewidth]{./fig/linr_pc_th.png}
    \caption{Execution time breakdown for Page Cache Linear Regression TeraHeap Spark investigation.}
    \label{fig:linr_pc_th}

    \includegraphics[width=\linewidth]{./fig/pr_h1_th.png}
    \caption{Execution time breakdown for H1 Page Rank TeraHeap Spark
    investigation.} 
    \label{fig:pr_h1_th}
\end{figure}

\begin{figure}[thbp]
    \includegraphics[width=\linewidth]{./fig/pr_pc_th.png}
    \caption{Execution time breakdown for PageCache Page Rank TeraHeap
    Spark investigation.}
    \label{fig:pr_pc_th}

    \includegraphics[width=\linewidth]{./fig/pr_h1_native.png}
    \caption{Execution time breakdown for PageCache Page Rank TeraHeap
    Spark investigation.}
    \label{fig:pr_h1_native}
\end{figure}

\begin{figure}[thbp]
    \includegraphics[width=\linewidth]{./fig/rw_pr_pc_th.png}
    \caption{Read-Write traffic over time for PageCache Page Rank
    TeraHeap Spark investigation.}
    \label{fig:rw_pr_pc_th}

    \includegraphics[width=\linewidth]{./fig/rw_linr_h1_th.png}
    \caption{Read-Write traffic over time for H1 Linear Regression
    TeraHeap Spark investigation.}
    \label{fig:rw_linr_h1_th}
\end{figure}

\begin{figure}[thbp]
    \includegraphics[width=\linewidth]{./fig/rw_linr_pc_th.png}
    \caption{Read-Write traffic over time for PageCache Linear
    Regression TeraHeap Spark investigation.}
    \label{fig:rw_linr_pc_th}

    \includegraphics[width=\linewidth]{./fig/rw_pr_h1_th.png}
    \caption{Read-Write traffic over time for H1 Page Rank TeraHeap
    Spark investigation.}
    \label{fig:rw_pr_h1_th}
\end{figure}

\subsection{Java Heap and I/O cache investigation}

Spark's memory management is critical for the performance of big data
analytics applications. In Spark, memory is divided into two
regions: execution memory, and storage memory. Execution memory
is used for storing data during shuffle and join operations and for
caching frequently accessed data. Storage memory is used for storing long lived
cached data. Additionally, Spark uses a combination
of in-memory and disk-based storage to provide efficient data access.
Spark provides various storage levels, including MEMORY-ONLY,
MEMORY-AND-DISK, and DISK-ONLY, to allow users to balance between
memory usage and data availability. We choose MEMORY-AND-DISK for Native Spark 
to cache 50\% of the RDDs in memory and 50\%s off-heap, in the storage device,
to balance memory and storage usage. Spark needs significant amounts
of memory even with the use of an off-heap compute cache. Spark uses
direct I/O to cache RDDs off-heap, while TeraHeap uses MMIO and is in need
of the Linux PageCache. Since our spark applications run within a
memory-limited cgroup in order to assure fair performance in-between
instances, that means that we have to investigate how the different
Spark workloads are performing with different amounts of H1 (Java Heap) for Native-TeraHeap Spark and Giraph and 
and of I/O cache for TeraHeap Spark-Giraph.
Increasing/decreasing H1 automaticaly does the opposite to the I/O
cache, because of the cgroup memory limit. 
Doing this investigation helps us run the colocated experiments using appropriate memory budget
configurations. Moreover we confirm that using a heap amount of 80\% to total DRAM for a group of processes
is the right baseline.
We break the execution time to Major GC, Minor GC and Other time which includes time going to IO from applications threads.

Figure \ref{fig:pr_h1_native} shows performance of single-instance Native Spark
running PageRank with adjustable size for H1 while available DRAM for the rest of the services is kept
steady at 16 GB. This graph shows that decreasing the size of H1
indicates a significant increase to Major GC.
Figure \ref{fig:gcs_pr_h1_native} justifies these claims by showing the obvious
decrease of the number of major gcs.

Figure \ref{fig:pr_h1_th} shows performance of single-instance TeraHeap Spark
running PageRank with adjustable size for H1 while avalaible DRAM for Page Cache is kept
steady at 16 GB. This graph shows that decreasing the size of H1
indicates a significant increase to Minor GC and a slight increase to
Major GC. Figure \ref{fig:gcs_pr_h1_th} justifies these claims by showing the obvious
decrease of the number of minor gcs. 
Figure \ref{fig:pr_pc_th} shows performance of
single-instance TeraHeap Spark running PageRank with adjustable size
for Page Cache while H1 is kept steady at 44 GB. This graph shows that
decreasing the size of PageCache indicates no changes to other time.
Figure \ref{fig:rw_pr_pc_th} justifies these claims by showing read-traffic to be steady.

Figure \ref{fig:linr_h1_native} shows performance of single-instance
Native Spark running LinearRegression with adjustable size for H1
while PageCache is kept steady at 8 GB. This graph shows that
decreasing the size of H1 indicates a significant increase to Major GC
and a slight increase to S/D. Other time remains the same.
Figure \ref{fig:gcs_linr_h1_native} by showing the number of minor-major gcs to decrease
while H1 increases (gc time). 

Figure \ref{fig:linr_h1_th} shows performance of
single-instance TeraHeap Spark running LinearRegression with
adjustable size for H1 while PageCache is kept steady at 8 GB. This
graph shows that decreasing the size of H1 indicates no increase to
GC. Figure \ref{fig:gcs_linr_h1_th} and \ref{fig:rw_linr_h1_th}
justify these numbers by showing the number of major gcs to stay the
same while H1 increases (gc time) and the read-write traffic to remain
steady (other). 
Figure \ref{fig:linr_pc_th} shows performance of
single-instance TeraHeap Spark running LinearRegression with
adjustable size for PageCache while H1 is kept steady at 28 GB. This
graph shows that decreasing the size of PageCache indicates no changes
to Other time. So changes to PageCache do not affect this workload.
Figure \ref{fig:gcs_linr_pc_th} and \ref{fig:rw_linr_pc_th} justify
these numbers by showing the number of minor-major gcs as well as
read-write traffic to remain the same.

\iffalse
\subsection{Metrics} 
When measuring performance, it's
important to choose metrics that provide a comprehensive view of the
system's behavior. In the case of measuring the performance of Spark and Giraph
colocated instances, there are several key metrics that one should consider.
These include heap capacity, which is the amount of memory allocated
GC time is also an
important metric, as it measures the amount of time spent by the JVM
garbage collector in freeing up memory. Serialization/deserialization
time, measured using Java async-profiler \cite{Profiler}, is important for
understanding how much time is spent in this operation, which can be a
bottleneck for some workloads. Other time, which is simply the
difference between total time and GC and serialization/deserialization
time, can provide insight into other factors that may be affecting
performance, but mainly includes the time spent in I/O and also the
time spent by mutator threads to run the application code. Device
traffic, measured using iostat, is important for understanding how
much data is being read from and written to storage devices. CPU idle
and IO wait, measured using mpstat, can help identify how much of the
CPU and I/O resources are being utilized. Finally, average throughput,
measured using Spark Bench, is a good indicator of the overall
performance of the system. Other metrics, such as the total amount of
data processed and the number of minor and major garbage collections,
as measured using jstat, can also provide valuable insights into
system behavior. By considering a range of metrics, one can get a more
accurate and comprehensive view of the performance of Spark instances.
\fi

\subsection{Benefits of colocating instances} 
Concurrent execution of workloads provides several benefits.
Firstly, it enables optimal resource utilization by effectively
leveraging the available hardware resources, including CPUs, memory,
and storage. Rather than leaving server resources idle between
workloads, concurrent execution ensures their efficient utilization,
leading to improved throughput and enhanced server efficiency.
Additionally, the consolidation of multiple workloads onto a single
server reduces hardware footprint, simplifies management, and
minimizes operational costs associated with managing multiple servers.

Another advantage is the potential for increased throughput. By
executing multiple workloads concurrently, tasks progress
simultaneously, resulting in faster completion and higher overall
throughput. This approach is particularly valuable when workloads
exhibit varying levels of computational intensity or have different
resource requirements. Concurrent execution allows for efficient
resource allocation, enabling each workload to access the necessary
resources and perform optimally.

Concurrent execution also facilitates workload prioritization,
allowing organizations to allocate resources based on workload
importance or urgency. By running multiple workloads concurrently,
critical or high-priority tasks can be assigned the required resources
and processed in a timely manner. This flexibility in resource
allocation and workload prioritization ensures efficient utilization
of available resources and improves overall performance.

Furthermore, the concurrent execution of workloads supports
experimentation and testing. By running workloads concurrently on the
same server, comparisons, performance benchmarking, and optimization
can be performed in a controlled environment. This concurrent
execution environment enables organizations to evaluate and fine-tune
applications effectively.

Moreover, concurrent execution of workloads allows users to run
workloads along with other users simultaneously. While this is a
method that wouldn't be considered correct in order to evaluate
something because of the interference of other applications running
concurrently, it is something common for the cloud and datacenters
where companies-users share the cloud-server.

Scalability is another advantage of concurrent execution. As data
volumes and processing demands increase, running multiple workloads
concurrently allows for horizontal scalability. Additional 
worker nodes can be added to accommodate larger workloads or handle
additional workloads without the need for significant infrastructure
changes. This scalability ensures that the system can handle growing
demands while maintaining high performance.

In conclusion, the concurrent execution of multiple workloads on
a single server offers significant advantages for performance
optimization. It enables optimal resource utilization, workload
consolidation, improved throughput, workload prioritization,
experimentation, and scalability. By carefully managing resources,
workload scheduling, and monitoring, organizations can achieve higher
performance, reduce infrastructure costs, and simplify management.

\subsection{Cost estimation}
Renting servers is a common practice for organizations requiring
computational resources, and the question arises as to whether
reducing the monetary cost is possible by achieving higher throughput
and faster workload completion. The relationship between cost
reduction and achieving higher throughput on rented servers is indeed
significant. By optimizing server performance, efficiently utilizing
resources, implementing workload scheduling, and improving
productivity, organizations can realize cost savings. Achieving higher
throughput and faster workload completion can lead to a reduced rental
duration, minimizing the time and associated costs of server usage.
Efficient resource utilization and workload scheduling contribute to
cost reduction by minimizing the number of servers required and
maximizing their utilization. Rental pricing models that take into
account resource utilization or data processed can further reduce
costs for organizations achieving higher throughput. Additionally,
improved productivity resulting from higher throughput and faster
workload completion enhances overall efficiency, allowing
organizations to accomplish more work within the same rental period
and reducing rental expenses. Therefore, pursuing higher throughput
and faster workload completion offers tangible benefits in terms of
monetary cost reduction for organizations renting servers. 

In order to estimate the cost of our evaluation in real-world public clusters we
chose a variety of providers like Amazon \cite{EC2}, Google \cite{GCP} and Microsoft \cite{Azure}. This
way we covered the most known providers and platforms someone would
choose to run their workloads on. We chose 2 machines from each
platform identical to the specifications of our 64, 128 and 256 GB DRAM
machines. These are the cheapest machines of that particular category
offered by the platform. We then used the platform's pricing
calculator to estimate the cost of renting that machine for the time
needed for each configuration to finish execution of all instances. Finally, we noticed that the price for
renting the storage device is really amenable to the cost for renting the machine.

\section{Evaluation}
\label{sec:eval}

\begin{figure}[htbp]
\centering
\begin{subfigure}[b]{0.48\textwidth}
    \includegraphics[width=\linewidth]{./fig/linr64.png}
    \caption{Execution time breakdown for multiple instances of
    LinearRegression using the 64 GB total DRAM setup. E.g.
    22-28-64-Native 1 indicates the first of the 2-4-8 instances that
    was run in parallel and uses 22 GB H1 - 28 GB total cgroup DRAM
    and 64 GB total DRAM for the machine.}
    \label{fig:linr64}
\end{subfigure}
\begin{subfigure}[b]{0.48\textwidth}
    \includegraphics[width=\linewidth]{./fig/pr64.png}
    \caption{Execution time breakdown for multiple instances of
    PageRank using the 64 GB total DRAM setup. E.g. 22-28-64-Native 1
    indicates the first of the 2-4-8 instances that was run in
    parallel and uses 22 GB H1 - 28 GB total cgroup DRAM and 64 GB
    total DRAM for the machine.}
    \label{fig:pr64}
\end{subfigure}\\[1em]
\end{figure}

\begin{figure}[htbp]
	\centering
        \begin{subfigure}[b]{0.48\textwidth}
    \includegraphics[width=\linewidth]{./fig/logr64.png}
    \caption{Execution time breakdown for multiple instances of
    Logistic Regression using the 64 GB total DRAM setup. E.g.
    22-28-64-Native 1 indicates the first of the 2-4-8 instances that
    was run in parallel and uses 22 GB H1 - 28 GB total cgroup DRAM
    and 64 GB total DRAM for the machine.}
    \label{fig:logr64}
\end{subfigure}

\begin{subfigure}[b]{0.48\textwidth}
    \includegraphics[width=\linewidth]{./fig/cc64.png}
    \caption{Execution time breakdown for multiple instances of
    Connected Component using the 64 GB total DRAM setup. E.g.
    22-28-64-Native 1 indicates the first of the 2-4-8 instances that
    was run in parallel and uses 22 GB H1 - 28 GB total cgroup DRAM
    and 64 GB total DRAM for the machine.}
    \label{fig:cc64}
\end{subfigure}\\[1em]
\end{figure}

\begin{figure}[ht!]
    \includegraphics[width=\linewidth]{./fig/pr256.png}
    \caption{Execution time breakdown for multiple instances of
    PageRank using the 256 GB total DRAM setup. E.g. 44-60-256-Native
    1 indicates the first of the 2-4-8 instances that was run in
    parallel and uses 44 GB H1 - 60 GB total cgroup DRAM and 256 GB
    total DRAM for the machine.} 
    \label{fig:pr256}
\end{figure}


\subsection{Experiments with colocated instances}

Here we look at the colocated experiments of Spark and Giraph.
We explain each figure from 2 aspects:
\begin{itemize}
\item{The differences in the time breakdown while number of instances increase for each configuration.}
\item{A comparison between the different configurations while instances increase.}
\end{itemize}

Figure \ref{fig:linr64} shows the performance of multiple
Native-TeraHeap Spark instances running LinearRegression with 64 GB
dataset per instance in our 64 GB DRAM machine. Each instance of Spark
uses one executor with 8 cores per executor. Available DRAM is 56 GB
and 8 GB are left to the Operating system, resulting in 64 GB total
DRAM. All configurations utilize 56 of 64 GB total DRAM. 
Starting from the left of the graph, the first 6 bars show the
performance of 3 runs. The first run is with 2 colocated Native Spark instances.
Another run with 2 colocated TH Spark instances with H1 dominating Page Cache
and a third run with 2 colocated TH Spark instances where Page Cache dominates H1. 
Each instance of the first 2 runs uses 22 GB DRAM for H1 (Java Heap) and 6 GB for rest of the services.
The third run uses 11 GB DRAM for H1 and 17 GB for Page Cache for each instance. 
The rest 12 bars show the performance of another 3 runs. The first run is with 4 colocated Native Spark instances.
Another run with 4 colocated TH Spark instances with H1 dominating Page Cache
and a third run with 4 colocated TH Spark instances where Page Cache dominates H1. 
Each instance of the first run uses 11 GB DRAM for H1 (Java Heap) and 3 GB for rest of the services. 
The second run uses 10 GB DRAM for H1 and 4 GB for Page Cache for each instance. 
The third run uses 6 GB DRAM for H1 and 8 GB for Page Cache for each instance. 
Considering the first aspect we see that GC and S/D increase dramatically for Native Spark along with significant increase to Other time. GC differences are witnessed because the heap capacity decreases and that causes memory pressure. TeraHeap Spark shows a slight increase to Major GC while the number of instances increases. This is because of the decreased heap capacity. Other time increases because more objects are moved to TeraHeap and read/write traffic increases but this is a good trade-off because all the GC is absorbed. S/D is completely absorbed by MMIO. From the third aspect, as instances increase in the server the benefit gap between Native and TeraHeap Spark becomes bigger. As Native Spark starves from more GC and S/D, TeraHeap maintains its benefits. 

Figure \ref{fig:pr64} shows the performance of multiple
Native-TeraHeap Spark instances running PageRank with 8 GB
dataset per instance in our 64 GB DRAM machine. Each instance of Spark
uses one executor with 8 cores per executor. Available DRAM is 56 GB
and 8 GB are left to the Operating system, resulting in 64 GB total
DRAM. All configurations utilize 56 of 64 GB total DRAM.
Starting from the left of the graph, the first 6 bars show the
performance of 3 runs. The first run is with 2 colocated Native Spark instances.
Another run with 2 colocated TH Spark instances with H1 dominating Page Cache
and a third run with 2 colocated TH Spark instances where Page Cache dominates H1.
Each instance of the first 2 runs uses 22 GB DRAM for H1 (Java Heap) and 6 GB for rest of the services.
The third run uses 11 GB DRAM for H1 and 17 GB for Page Cache for each instance. 
The next 12 bars show the performance of another 3 runs. The first run is with 4 colocated Native Spark instances.
Another run with 4 colocated TH Spark instances with H1 dominating Page Cache
and a third run with 4 colocated TH Spark instances where Page Cache dominates H1.
Each instance of the first run uses 11 GB DRAM for H1 (Java Heap) and 3 GB for rest of the services.
The second run uses 10 GB DRAM for H1 and 4 GB for Page Cache for each instance.
The third run uses 6 GB DRAM for H1 and 8 GB for Page Cache for each instance.
The last 8 bars refer to 8 colocated instances of TeraHeap Spark only. 
We were unable to decrease H1 enough to run 8 colocates instance of Native Spark
because JVM runs out of memory. Each instance of the run uses 4 GB DRAM for H1 (Java Heap) and 3 GB for Page Cache.
Considering the first aspect we see that Minor and Major GC increase dramatically for Native Spark along with significant increase to Other time. Minor and Major GC differences are witnessed because the heap capacity decreases and that causes memory pressure. TeraHeap Spark shows a slight increase to Major GC while the number of instances increases. This is because of the decreasing heap capacity. Other time increases because more objects are moved to TeraHeap but this is a good trade-off because all the GC is absorbed. S/D is completely absorbed by MMIO. From the second aspect, as instances increase in the server the benefit gap between Native and TeraHeap Spark becomes bigger. As Native Spark starves from more GC and S/D, TeraHeap maintains its benefits.

Figure \ref{fig:logr64} shows the performance of multiple
Native-TeraHeap Spark instances running Logistic Regression with 64 GB
dataset per instance in our 64 GB DRAM machine. Each instance of Spark
uses one executor with 8 cores per executor. Available DRAM is 56 GB
and 8 GB are left to the Operating system, resulting in 64 GB total
DRAM. All configurations utilize 56 of 64 GB total DRAM.
Starting from the left of the graph, the first 6 bars show the
performance of 3 runs. The first run is with 2 colocated Native Spark instances.
Another run with 2 colocated TH Spark instances with H1 dominating Page Cache
and a third run with 2 colocated TH Spark instances where Page Cache dominates H1.
Each instance of the first 2 runs uses 22 GB DRAM for H1 (Java Heap) and 6 GB for rest of the services.
The third run uses 11 GB DRAM for H1 and 17 GB for Page Cache for each instance. 
The next 12 bars show the performance of another 3 runs. The first run is with 4 colocated Native Spark instances.
Another run with 4 colocated TH Spark instances with H1 dominating Page Cache
and a third run with 4 colocated TH Spark instances where Page Cache dominates H1.
Each instance of the first run uses 11 GB DRAM for H1 (Java Heap) and 3 GB for rest of the services.
The second run uses 10 GB DRAM for H1 and 4 GB for Page Cache for each instance.
The third run uses 6 GB DRAM for H1 and 8 GB for Page Cache for each instance.
Considering the first aspect we see that GC and S/D increase dramatically for Native Spark along with significant increase to Other time. GC differences are witnessed because the heap capacity decreases and that causes memory pressure. TeraHeap Spark shows a slight increase to Major GC while the number of instances increases. This is because of the decreased heap capacity. Other time increases because more objects are moved to TeraHeap and read/write traffic increases but this is a good trade-off because all the GC is absorbed. S/D is completely absorbed by MMIO. From the third aspect, as instances increase in the server the benefit gap between Native and TeraHeap Spark becomes bigger. As Native Spark starves from more GC and S/D, TeraHeap maintains its benefits. 

Figure \ref{fig:cc64} shows the performance of multiple
Native-TeraHeap Spark instances running Connected Component with 8 GB
dataset per instance in our 64 GB DRAM machine. Each instance of Spark
uses one executor with 8 cores per executor. Available DRAM is 56 GB
and 8 GB are left to the Operating system, resulting in 64 GB total
DRAM. All configurations utilize 56 of 64 GB total DRAM.
Starting from the left of the graph, the first 6 bars show the
performance of 3 runs. The first run is with 2 colocated Native Spark instances.
Another run with 2 colocated TH Spark instances with H1 dominating Page Cache
and a third run with 2 colocated TH Spark instances where Page Cache dominates H1.
Each instance of the first 2 runs uses 22 GB DRAM for H1 (Java Heap) and 6 GB for rest of the services.
The third run uses 11 GB DRAM for H1 and 17 GB for Page Cache for each instance. 
The next 12 bars show the performance of another 3 runs. The first run is with 4 colocated Native Spark instances.
Another run with 4 colocated TH Spark instances with H1 dominating Page Cache
and a third run with 4 colocated TH Spark instances where Page Cache dominates H1.
Each instance of the first run uses 11 GB DRAM for H1 (Java Heap) and 3 GB for rest of the services.
The second run uses 10 GB DRAM for H1 and 4 GB for Page Cache for each instance.
The third run uses 6 GB DRAM for H1 and 8 GB for Page Cache for each instance.
Considering the first aspect we see that Minor and Major GC increase dramatically for Native Spark along with significant increase to Other time. Minor and Major GC differences are witnessed because the heap capacity decreases and that causes memory pressure. TeraHeap Spark shows a slight increase to Major GC while the number of instances increases. This is because of the decreasing heap capacity. Other time increases because more objects are moved to TeraHeap but this is a good trade-off because all the GC is absorbed. S/D is completely absorbed by MMIO. From the second aspect, as instances increase in the server the benefit gap between Native and TeraHeap Spark becomes bigger. As Native Spark starves from more GC and S/D, TeraHeap maintains its benefits.

Figure \ref{fig:pr256} shows the performance of multiple
Native-TeraHeap Spark instances running PageRank with 32 GB
dataset per instance in our 256 GB DRAM machine. Each instance of Spark
uses one executor with 8 cores per executor. Available DRAM is 240 GB
and 16 GB are left to the Operating system, resulting in 256 GB total
DRAM. All configurations utilize 240 of 256 GB total DRAM.
Starting from the left of the graph, the first 8 bars show the
performance of 2 runs. The first run is with 4 colocated Native Spark instances.
Another run with 4 colocated TH Spark instances with H1 dominating Page Cache.
Each instance of the first 8 runs uses 48 GB DRAM for H1 (Java Heap) and 12 GB for rest of the services including Page Cache.
The next 21 bars show the performance of another 3 runs. The first run is with 7 colocated Native Spark instances.
Another run with 7 colocated TH Spark instances with H1 dominating Page Cache
and a third run with 7 colocated TH Spark instances where Page Cache dominates H1.
Each instance of the first run uses 27 GB DRAM for H1 (Java Heap) and 7 GB for rest of the services.
The second run uses 27 GB DRAM for H1 and 7 GB for Page Cache for each TH instance.
The third run uses 14 GB DRAM for H1 and 20 GB for Page Cache for each TH instance.
Considering the first aspect we see that Minor and Major GC increase dramatically for Native Spark along with significant increase to Other time. Minor and Major GC differences are witnessed because the heap capacity decreases and that causes memory pressure. TeraHeap Spark shows a slight increase to Major GC while the number of instances increases. This is because of the decreasing heap capacity. Other time increases because more objects are moved to TeraHeap but this is a good trade-off because all the GC is absorbed. S/D is completely absorbed by MMIO. From the second aspect, as instances increase in the server the benefit gap between Native and TeraHeap Spark becomes bigger. As Native Spark starves from more GC and S/D, TeraHeap maintains its benefits.

\subsection{Is the CPU utilization of the server increasing
accordingly to throughput?}
\begin{figure}[htbp]
	\centering
	\begin{subfigure}[b]{0.48\textwidth}
        \includegraphics[width=\linewidth]{./fig/PR_64_THR.png}
    \caption{Page Rank 64 GB DRAM setup Native and TeraHeap throughput
    as the number of instances increases. Configurations starting with
    N denote a run with Native instances of Spark and with T with
    TeraHeap. H1 is a run with the memory budget configured to contain
    a bigger size for H1 than PageCache and PC the opposite. E.g. T2
    PC is a run of 2 concurrent TeraHeap instances with exactly the
    same configuration.}
\label{fig:pr_64_thr}
\end{subfigure}
        \begin{subfigure}[b]{0.48\textwidth}
        \includegraphics[width=\linewidth]{./fig/PR_64_USR.png}
    \caption{Page Rank 64 GB DRAM setup Native and TeraHeap User CPU utilization
    as the number of instances increases. Configurations starting with
    N denote a run with Native instances of Spark and with T with
    TeraHeap. H1 is a run with the memory budget configured to contain
    a bigger size for H1 than PageCache and PC the opposite. E.g. T2
    PC is a run of 2 concurrent TeraHeap instances with exactly the
    same configuration.}
		\label{fig:pr_64_usr}
	\end{subfigure}\\[1em]
\end{figure}

\begin{figure}[htbp]
	\centering
        \begin{subfigure}[b]{0.48\textwidth}
        \includegraphics[width=\linewidth]{./fig/LINR_64_THR.png}
    \caption{Linear Regression 64 GB DRAM setup Native and TeraHeap
    throughput as the number of instances increases. Configurations
    starting with N denote a run with Native instances of Spark and
    with T with TeraHeap. H1 is a run with the memory budget
    configured to contain a bigger size for H1 than PageCache and PC
    the opposite. E.g. T2 PC is a run of 2 concurrent TeraHeap
    instances with exactly the same configuration.}
		\label{fig:linr_64_thr}
        \end{subfigure}
        \begin{subfigure}[b]{0.48\textwidth}
        \includegraphics[width=\linewidth]{./fig/LOGR_64_THR.png}
    \caption{Logistic Regression 64 GB DRAM setup Native and TeraHeap
    throughput as the number of instances increases. Configurations
    starting with N denote a run with Native instances of Spark and
    with T with TeraHeap. H1 is a run with the memory budget
    configured to contain a bigger size for H1 than PageCache and PC
    the opposite. E.g. T2 PC is a run of 2 concurrent TeraHeap
    instances with exactly the same configuration.}
		\label{fig:logr_64_thr}
        \end{subfigure}\\[1em]
\end{figure}

\begin{figure}[htbp]
	\centering
        \begin{subfigure}[b]{0.48\textwidth}
        \includegraphics[width=\linewidth]{./fig/LINR_64_USR.png}
    \caption{Linear Regression 64 GB DRAM setup Native and TeraHeap
    User CPU utilization as the number of instances increases. Configurations
    starting with N denote a run with Native instances of Spark and
    with T with TeraHeap. H1 is a run with the memory budget
    configured to contain a bigger size for H1 than PageCache and PC
    the opposite. E.g. T2 PC is a run of 2 concurrent TeraHeap
    instances with exactly the same configuration.}
		\label{fig:linr_64_usr}
        \end{subfigure}
   \begin{subfigure}[b]{0.48\textwidth}
    \includegraphics[width=\linewidth]{./fig/LOGR_64_USR.png}
    \caption{Logistic Regression 64 GB DRAM setup Native and TeraHeap
    User CPU utilization as the number of instances increases. Configurations
    starting with N denote a run with Native instances of Spark and
    with T with TeraHeap. H1 is a run with the memory budget
    configured to contain a bigger size for H1 than PageCache and PC
    the opposite. E.g. T2 PC is a run of 2 concurrent TeraHeap
    instances with exactly the same configuration.}
	   \label{fig:logr_64_usr}
    \end{subfigure}\\[1em]
\end{figure}

\begin{figure}[htbp]
	\centering
        \begin{subfigure}[b]{0.48\textwidth}
        \includegraphics[width=\linewidth]{./fig/CC_64_THR.png}
    \caption{Connected Component 64 GB DRAM setup Native and TeraHeap
    throughput as the number of instances increases.Configurations
    starting with N denote a run with Native instances of Spark and
    with T with TeraHeap. H1 is a run with the memory budget
    configured to contain a bigger size for H1 than PageCache and PC
    the opposite. E.g. T2 PC is a run of 2 concurrent TeraHeap
    instances with exactly the same configuration. }
		\label{fig:cc_64_thr}
	\end{subfigure}
\begin{subfigure}[b]{0.48\textwidth}
        \includegraphics[width=\linewidth]{./fig/CC_64_USR.png}
    \caption{Connected Component 64 GB DRAM setup Native and TeraHeap
    User CPU utilization as the number of instances increases. Configurations
    starting with N denote a run with Native instances of Spark and
    with T with TeraHeap. H1 is a run with the memory budget
    configured to contain a bigger size for H1 than PageCache and PC
    the opposite. E.g. T2 PC is a run of 2 concurrent TeraHeap
    instances with exactly the same configuration.}
	\label{fig:cc_64_usr}
\end{subfigure}\\[1em]
\end{figure}


The main goal for colocating tasks is to increase the CPU utilization and achieve better
throughput. CPU utilization is split to 2 parts. 
User utilization includes all CPU cycles that were executed in user-space threads.
It includes GC cycles, S/D cycles and mutator tasks except I/O.
System utilization includes all CPU cycles that were executed in kernel-space threads.
This includes I/O carried out by GC (TeraHeap) and mutator I/O.
Therefore we have to focus to User utilization which includes the effective CPU cycles done by the application.
By looking at figures \ref{fig:pr_64_thr}, \ref{fig:linr_64_thr},
\ref{fig:logr_64_thr} and \ref{fig:cc_64_thr} we see that Native
Spark's throughput decreases as the number of colocated
instances-executors increase in the server.
This is justified by GC and S/D that increase as instances increase and by the increased User utilization
(\ref{fig:pr_64_usr}, \ref{fig:linr_64_usr}, \ref{fig:logr_64_usr}, \ref{fig:cc_64_usr}).
TeraHeap achieves higher throughput than Native Spark and that is also justified by the increased User utilization.
Having higher user utilization and lower GC and S/D means that effective CPU utilization is really higher, since more
work is done by mutator threads.

\subsection{Single instance vs colocated instances}
When running colocated instances the goal is that the execution 
of all instances is close to the summary of execution of all instances run isolated.
As seen in tables, most of the times we are not even close. To achieve that
we need a scheduling policy to dynamically resize H1 and Page Cache based on runtime decisions.

\subsection{What
happens with monetary cost across different cloud platforms?}

\iffalse
\begin{table}[htbp]
  \centering
	\begin{subtable}[b]{0.45\linewidth}
  \caption{Page Rank synopsis table. Configurations starting
    with N denote a run with Native instances of Spark and with T with
    TeraHeap. H1 is a run with the memory budget configured to contain
    a bigger size for H1 than PageCache and PC the opposite.}
  \label{tab:pr_table}
        %\resizebox{19cm}{!}{
        \begin{tabular}{|c|c|c|c|c|c|c|c|c|c|c|c|c|}
      \hline
\textbf{Conf.} & \textbf{H1 Size/I} & \textbf{Memory/I} & \textbf{Total Mem.} & \textbf{\#I} & \textbf{Exec. Time} & \textbf{CPU Idle} & \textbf{Total MB Proc.} & \textbf{MB/s} & \textbf{MB/s/I} & \textbf{Cost AWS \$} & \textbf{Cost GCP \$} & \textbf{Cost Azure \$} \\
        \hline
    N2 - small & 22 & 28 & 64 & 2 & 3563 & 29 & 16980 & 5 & 2 & 0.6 & 0.58 & 0.67 \\
    N4 - small & 11 & 14 & 64 & 4 & 8195 & 11 & 33960 & 4 & 1 & 1.8 & 1.74 & 2.01 \\
    T2 H1 – small & 22 & 28 & 64 & 2 & 2545 & 7 & 16980 & 7 & 3 & 0.6 & 0.58 & 0.67 \\
    T2 PC – small & 11 & 28 & 64 & 2 & 2385 & 9 & 16980 & 7 & 4 & 0.6 & 0.58 & 0.67 \\
    T4 H1 – small & 11 & 14 & 64 & 4 & 5554 & 1 & 33960 & 6 & 2 & 1.2 & 1.16 & 1.34 \\
    T4 PC – small & 6 & 14 & 64 & 4 & 5880 & 1 & 33960 & 6 & 2 & 1.2 & 1.16 & 1.34 \\
    N8 – small & 4 & 7 & 64 & 8 & OOM & ** & 0 & 0 & 0 & *** & *** & *** \\
    T8 – small & 4 & 7 & 64 & 8 & 12305 & 0 & 67920 & 6 & 1 & 2.4 & 2.32 & 2.68 \\
    N4 - big & 44 & 60 & 256 & 4 & 13542 & 15 & 135872 & 10 & 3 & 6.4 & *** & *** \\
    T4 – big & 48 & 60 & 256 & 4 & 9284 & 5 & 135782 & 16 & 4 \\      
	\hline
     \end{tabular}%
        %}
\end{subtable}

	\begin{subtable}[b]{0.45\linewidth}
  \caption{Linear Regression synopsis table. Configurations starting
    with N denote a run with Native instances of Spark and with T with
    TeraHeap. H1 is a run with the memory budget configured to contain
    a bigger size for H1 than PageCache and PC the opposite.}
  \label{tab:linr_table}
        %\resizebox{19cm}{!}{
        \begin{tabular}{|c|c|c|c|c|c|c|c|c|c|c|c|c|}
      \hline
\textbf{Conf.} & \textbf{H1 Size/I} & \textbf{Memory/I} & \textbf{Total Mem.} & \textbf{\#I} & \textbf{Exec. Time} & \textbf{CPU Idle} & \textbf{Total MB Proc.} & \textbf{MB/s} & \textbf{MB/s/I} & \textbf{Cost AWS \$} & \textbf{Cost GCP \$} & \textbf{Cost Azure \$} \\
	\hline 
      N2 & 22 & 28 & 64 & 2 & 3745 & 20 & 134896 & 37 & 18 & 0.6 & 0.58 & 0.67 \\ 
      N4 & 11 & 14 & 64 & 4 & 13874 & 7 & 269792 & 20 & 5 & 2.4 & 2.32 & 2.01 \\
      T2 H1 & 22 & 28 & 64 & 2 & 2891 & 19 & 134896 & 48 & 24 & 0.6 & 0.58 & 0.67 \\
      T2 PC & 11 & 28 & 64 & 2 & 2747 & 18 & 134896 & 49 & 25 & 0.6 & 0.58 & 0.67 \\
      T4 H1 & 11 & 14 & 64 & 4 & 6075 & 2 & 269792 & 44 & 11 & 1.2 & 1.16 & 1.34 \\
      T4 PC & 6 & 14 & 64 & 4 & 6176 & 3 & 269792 & 44 & 11 & 1.2 & 1.16 & 1.34 \\ 
      \hline
     \end{tabular}%
	%}
\end{subtable}
	\vspace{1em}
\end{table}
\fi
\iffalse
%\begin{center}%[htbp]
 %   \centering
  %  \caption{Page Rank synopsis table}
	%\begin{tabular}{|c|c|c|c|c|c|c|c|c|c|c|c|c|c|c|c|c|}
		\begin{tabularx}{\linewidth}{*{17}{X}}
	    \hline
        Configuration & H1 Size / I & Memory / I & Total memory & \#I & Exec. Time & User util. & System util. & IO Wait & CPU Idle & Total MB Processed & MB/s & MB/s/I & Cost AWS \$ & Cost GCP \$ & Cost Azure \$ \\
        \hline
        N2-small – single & 22 & 28 & 64 & 1 & 1762 & 32 & 2 & 1 & 65 & 8490 & 5 & 5 & 0.6 & 0.58 & 0.67 \\
        N2 - small & 22 & 28 & 64 & 2 & 3563 & 66 & 5 & 2 & 27 & 16980 & 5 & 2 & 0.6 & 0.58 & 0.67 \\
        N4 – small – single & 11 & 14 & 64 & 1 & 1783 & 29 & 2 & 1 & 68 & 8490 & 5 & 5 & 0.6 & 0.58 & 0.67 \\
        N4 - small & 11 & 14 & 64 & 4 & 8195 & 84 & 6 & 2 & 8 & 33960 & 4 & 1 & 1.8 & 1.74 & 2.01 \\
        N8 – small & 4 & 7 & 64 & 8 & OOM & 0 & 0 & 0 & ** & 0 & 0 & 0 & *** & *** \\
        T2 H1- small – single & 22 & 28 & 64 & 1 & 1146 & 46 & 2 & 1 & 51 & 8490 & 7 & 7 & 0.6 & 0.58 & 0.67 \\
        T2 H1 – small & 22 & 28 & 64 & 2 & 2545 & 89 & 4 & 1 & 6 & 16980 & 7 & 3 & 0.6 & 0.58 & 0.67 \\
        T2 PC- small – single & 11 & 28 & 64 & 1 & 966 & 45 & 2 & 1 & 52 & 8490 & 9 & 9 & 0.6 & 0.58 & 0.67 \\
        T2 PC – small & 11 & 28 & 64 & 2 & 2385 & 87 & 4 & 0 & 9 & 16980 & 7 & 4 & 0.6 & 0.58 & 0.67 \\
        T4 H1 – small – single & 11 & 14 & 64 & 1 & 984 & 43 & 3 & 1 & 53 & 8490 & 9 & 9 & 0.6 & 0.58 & 0.67 \\
        T4 H1 – small & 11 & 14 & 64 & 4 & 5554 & 94 & 5 & 0 & 1 & 33960 & 6 & 2 & 1.2 & 1.16 & 1.34 \\
        T4 PC – small – single & 6 & 14 & 64 & 1 & 922 & 42 & 2 & 2 & 54 & 8490 & 9 & 9 & 0.6 & 0.58 & 0.67 \\
        T4 PC – small & 6 & 14 & 64 & 4 & 5880 & 94 & 5 & 0 & 1 & 33960 & 6 & 2 & 1.2 & 1.16 & 1.34 \\
        T8 – small – single & 4 & 7 & 64 & 1 & 1037 & 39 & 2 & 1 & 58 & 8490 & 8 & 8 & 0.6 & 0.58 & 0.67 \\
        T8 – small & 4 & 7 & 64 & 8 & 12305 & 95 & 5 & 0 & 0 & 67920 & 6 & 1 & 2.4 & 2.32 & 2.68 \\
        N4 - big & 48 & 60 & 256 & 4 & 13542 & 78 & 7 & 7 & 8 & 135872 & 10 & 3 & 6.4 & *** & *** \\
        T4 – big & 48 & 60 & 256 & 4 & 9284 & 91 & 8 & 1 & 1 & 135782 & 16 & 4 & 1.8 & *** & *** \\
        N7 – big & 27 & 34 & 256 & 7 & 32763 & 80 & 8 & 12 & 0 & 237776 & 7 & 1 & 14.4 & *** & *** \\
        T7 H1 – big & 27 & 34 & 256 & 7 & 19443 & 92 & 7 & 1 & 0 & 237776 & 12 & 2 & 9.6 & *** & *** \\
        T7 PC – big & 14 & 34 & 256 & 7 & 16485 & 93 & 7 & 1 & 0 & 237776 & 14 & 2 & 8 & *** & *** \\
        \hline
	\bottomrule
    \end{tabularx}
%	\label{tab:pr_table}
%\end{center}
\fi
\iffalse
\begin{table}[t!]
  \centering
  \caption{Logistic Regression synopsis table. Configurations starting
    with N denote a run with Native instances of Spark and with T with
    TeraHeap. H1 is a run with the memory budget configured to contain
    a bigger size for H1 than PageCache and PC the opposite.}
  \label{tab:logr_table}
	\resizebox{19cm}{!}{
	\begin{tabular}{|c|c|c|c|c|c|c|c|c|c|c|c|c|}
      \hline
\textbf{Conf.} & \textbf{H1 Size/I} & \textbf{Memory/I} & \textbf{Total Mem.} & \textbf{\#I} & \textbf{Exec. Time} & \textbf{CPU Idle} & \textbf{Total MB Proc.} & \textbf{MB/s} & \textbf{MB/s/I} & \textbf{Cost AWS \$} & \textbf{Cost GCP \$} & \textbf{Cost Azure \$} \\
      \hline
      N2 & 22 & 28 & 64 & 2 & 5127 & 18 & 133348 & 26 & 13 & 1.2 & 1.16 & 0.67 \\
      N4 & 11 & 14 & 64 & 4 & 13730 & 7 & 266696 & 19 & 5 & 2.4 & 2.32 & 2.68 \\
      T2 H1 & 22 & 28 & 64 & 2 & 2861 & 18 & 133348 & 47 & 24 & 0.6 & 0.58 & 0.67 \\
      T2 PC & 11 & 28 & 64 & 2 & 2683 & 18 & 133348 & 50 & 25 & 0.6 & 0.58 & 0.67 \\
      T4 H1 & 10 & 14 & 64 & 4 & 5712 & 2 & 266696 & 47 & 12 & 1.2 & 1.16 & 1.34 \\
      T4 PC & 6 & 14 & 64 & 4 & 6138 & 2 & 266696 & 43 & 10 & 1.2 & 1.16 & 1.34 \\
      \hline
     \end{tabular}%
	}
\end{table}

\begin{table}[t!]
  \centering
  \caption{Connected Component synopsis table. Configurations starting
    with N denote a run with Native instances of Spark and with T with
    TeraHeap. H1 is a run with the memory budget configured to contain
    a bigger size for H1 than PageCache and PC the opposite.}
  \label{tab:cc_table}
        \resizebox{19cm}{!}{
        \begin{tabular}{|c|c|c|c|c|c|c|c|c|c|c|c|c|}
      \hline
\textbf{Conf.} & \textbf{H1 Size/I} & \textbf{Memory/I} & \textbf{Total Mem.} & \textbf{\#I} & \textbf{Exec. Time} & \textbf{CPU Idle} & \textbf{Total MB Proc.} & \textbf{MB/s} & \textbf{MB/s/I} & \textbf{Cost AWS \$} & \textbf{Cost GCP \$} & \textbf{Cost Azure \$} \\
	\hline
      N2 & 22 & 28 & 64 & 2 & 2958 & 26 & 16980 & 6 & 3 & 0.6 & 0.58 & 0.67 \\
      N4 & 11 & 14 & 64 & 4 & 7231 & 7 & 33960 & 5 & 3 & 1.8 & 1.74 & 2.01 \\
      T2 H1 & 22 & 28 & 64 & 2 & 2526 & 5 & 16980 & 7 & 4 & 0.6 & 0.58 & 0.67 \\
      T2 PC & 11 & 28 & 64 & 2 & 2519 & 7 & 16980 & 7 & 4 & 0.6 & 0.58 & 0.67 \\
      T4 H1 & 11 & 14 & 64 & 4 & 5439 & 3 & 33960 & 6 & 3 & 1.2 & 0.58 & 0.67 \\
      T4 PC & 6 & 14 & 64 & 4 & 5487 & 2 & 33960 & 6 & 3 & 1.2 & 1.16 & 1.34 \\
      \hline
     \end{tabular}%
	}
\end{table}
\fi

Tables \ref{tab:pr_table}, \ref{tab:linr_table}, \ref{tab:logr_table}
and \ref{tab:cc_table} show 
Amazon Web Services Cloud (EC2), GCP (Google Cloud Platform) and Microsoft Azure costs of deploying multiple instances of Spark
using both techniques. 
We witness that all providers offer a similar cost for identical machines to our server. 
As seen by the tables, taking into account that we have an hourly cost,
TeraHeap could be used for running colocated Spark and Giraph workloads
and save money. Reducing the GC and S/D makes a huge difference in the execution time
and therefore running with TeraHeap decreases the hours needed to rent the machines.


\section{Future Work}

While our proposed offloading technique shows promising results in
improving job throughput for big data analytics workloads on Spark
clusters, there are several avenues for future work to further improve
the performance and scalability of Spark clusters. 

Firstly, one potential direction for future work is to investigate the
use of other types of storage mediums such as the hybrid NVM. This
medium could improve the performance of Big data analytics further by
combining the advantages of memory and storage.

Secondly, another area for future work is to develop techniques for
dynamically adjusting the heap offloading decisions based on workload
characteristics and resource availability. For example, the offloading
decision can be based on the size of the input data or the
availability of DRAM capacity in the cluster. Such techniques can help
maximize the performance gains achieved by offloading while minimizing
the cost of offloading.

Thirdly, an interesting direction for future work is to explore the
use of heap offloading in environments where Spark clusters are
deployed across multiple machines using RDMA to achieve communication
between the different machines. This can help utilize the DRAM, CPU
and storage availability in more than one machine and provide a more
cost-effective solution for big data processing.

Finally, another potential area for future work is to investigate the
use of heap offloading for other big data processing frameworks beyond
Spark. Many other big data processing frameworks such as Apache Giraph
can potentially benefit from offloading techniques to improve their
performance and scalability.

Overall, there are many exciting avenues for future work in improving
the performance and scalability of big data processing frameworks such
as Spark. Our proposed offloading technique provides a solid
foundation for future work and offers a promising approach for
addressing the challenges of big data processing.

\section{Conclusions}

In this paper, we conducted an analysis of throughput for big data analytics
using a heap offloading technique that reduces memory pressure by moving parts of
the managed Java Heap to a secondary memory-mapped heap over fast
storage devices such as NVMe. We offered a detailed methodology on
how someone could use a system like TeraHeap to run Big Data Analytics workloads
to reduce memory pressure and therefore increase his server throughput by utilizng the CPU resources
in a more efficient way.

Our experimental results demonstrate the effectiveness of such a mechanism
using various Big data analytics workloads on a real-world cluster. We also
compare TeraHeap with the native Spark-Giraph distribution and show
that TeraHeap has the potential to further improve performance by reducing GC and eliminating S/D over data.

Overall, our analysis showed that using a system like TeraHeap offers a promising approach to
improving server throughput for big data analytics workloads, particularly for computation-intensive tasks. With the
increasing demand for efficient and scalable big data processing frameworks, this analysis provides a valuable contribution to the field
of big data analytics and memory management.


%-------------------------------------------------------------------------------
\bibliographystyle{plain}
\bibliography{paper}

%%%%%%%%%%%%%%%%%%%%%%%%%%%%%%%%%%%%%%%%%%%%%%%%%%%%%%%%%%%%%%%%%%%%%%%%%%%%%%%%
\end{document}
%%%%%%%%%%%%%%%%%%%%%%%%%%%%%%%%%%%%%%%%%%%%%%%%%%%%%%%%%%%%%%%%%%%%%%%%%%%%%%%%
