\section{Future Work}

While our proposed offloading technique shows promising results in
improving job throughput for big data analytics workloads on Spark
clusters, there are several avenues for future work to further improve
the performance and scalability of Spark clusters. 

Firstly, one potential direction for future work is to investigate the
use of other types of storage mediums such as the hybrid NVM. This
medium could improve the performance of Big data analytics further by
combining the advantages of memory and storage.

Secondly, another area for future work is to develop techniques for
dynamically adjusting the heap offloading decisions based on workload
characteristics and resource availability. For example, the offloading
decision can be based on the size of the input data or the
availability of DRAM capacity in the cluster. Such techniques can help
maximize the performance gains achieved by offloading while minimizing
the cost of offloading.

Thirdly, an interesting direction for future work is to explore the
use of heap offloading in environments where Spark clusters are
deployed across multiple machines using RDMA to achieve communication
between the different machines. This can help utilize the DRAM, CPU
and storage availability in more than one machine and provide a more
cost-effective solution for big data processing.

Finally, another potential area for future work is to investigate the
use of heap offloading for other big data processing frameworks beyond
Spark. Many other big data processing frameworks such as Apache Giraph
can potentially benefit from offloading techniques to improve their
performance and scalability.

Overall, there are many exciting avenues for future work in improving
the performance and scalability of big data processing frameworks such
as Spark. Our proposed offloading technique provides a solid
foundation for future work and offers a promising approach for
addressing the challenges of big data processing.
