\section{Future Work}

While this analysis shows promising results and provides a methodology for understanding
throughput for big data analytics workloads on Spark and Giraph
clusters, there are several avenues for future work to use it on and
improve performance and scalability. 

Firstly, one potential direction for future work is to investigate the
use of other types of storage mediums such as the hybrid NVM. This
medium could improve the performance of Big data analytics further by
combining the advantages of memory and storage.

Secondly, another area for future work is to develop techniques for
dynamically adjusting the heap offloading decisions based on workload
characteristics and resource availability. For example, the offloading
decision can be based on the size of the input data or the
availability of DRAM capacity in the cluster. Such techniques can help
maximize the performance gains achieved by offloading while minimizing
the cost of offloading.

Thirdly, an interesting direction for future work is to explore the
use of heap offloading in environments where Spark-Giraph clusters are
deployed across multiple machines using RDMA to achieve communication
between the different machines. This can help utilize the DRAM, CPU
and storage availability in more than one machine and provide a more
cost-effective solution for big data processing.

Finally, investigating the power consumption of our experiments would be 
very interesting, because we would examine the trade-offs between 
better performance and higher resource utilization with the cost in power.
