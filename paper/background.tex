\section{Background}
\label{sec:background}

In this section, we describe how TeraHeap eliminates GC and S/D.

TeraHeap is a system that eliminates S/D
and GC overheads for a large portion of the data in managed big data
analytics frameworks. TeraHeap extends the Java virtual machine
(JVM) to use a second, high-capacity heap (H2) over a fast storage
device that coexists alongside the regular heap (H1). It eliminates
S/D by providing direct access to objects in H2 and reduces GC
by avoiding costly GC scans over objects in H2. Frameworks use
TeraHeap through its hint-based interface without modifications to
the applications that run on top of them.
TeraHeap provides a hint-based interface that uses key-object opportunism
and enables frameworks to mark objects and indicate when to move
them to H2. During GC, TeraHeap starts from root key-objects and
dynamically identifies the objects to move to H2.

Furthermore, TeraHeap presents a unified heap
with the aggregate capacity of H1 and H2, where scans over H2
during GC are eliminated, to avoid expensive device I/O. To achieve
this, TeraHeap organizes H2 into regions with similar-lifetime objects. 
For space reclamation, the collector reclaims H1
objects as usual. For H2 regions, unlike existing region-based allocators, 
TeraHeap resolves the space-performance trade-off
for reclaiming space differently. Existing allocators reclaim region
space eagerly by moving live objects to another region, which would
generate excessive I/O for storage-backed regions. Instead, 
TeraHeap uses the high capacity of NVMe SSDs to reclaim entire regions
lazily, avoiding slow object compaction on the storage device.

