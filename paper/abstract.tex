\begin{abstract}
    Nowadays datacenters experience problems with scaling their DRAM
    capacity.  \note{jk: please rewrite this sentence - is very
    complex.} This limitation causes big data analytics frameworks,
    which run on top of the Java Virtual Machine (JVM), to experience
    lack of sufficient memory. The lack of sufficient memory leads to
    long garbage collection (GC) cycles during the data processing.
    \note{jk: Say that they use more memory to reduce this overhead
    and then conclude with the last sentence}
    This overhead prevents the frameworks from effectively utilizing
    the CPU thus leading to low server throughput.

    In this work, we conduct an analysis of server throughput for
    managed big data analytics frameworks when offloading the heap to
    fast storage devices. More specifically, we examine, whether
    storing data to fast storage devices instead of the main memory
    can help increase the throughput of the system by reducing memory
    pressure. \note{jk: we need to do make the term throughput more
    specific. E.g. Increase the CPU utilization} 
    We use a system called TeraHeap that moves objects from
    the Java managed heap to a secondary heap over a fast storage
    device to reduce the Garbage Collection overhead. We examine
    whether reducing the Java Garbage Collection overhead leads to
    more effective utilization of the available CPU by the
    application. Our primary focus is to analyze the system's
    performance under the colocation of multiple memory-bound
    instances.   Furthermore, we present a detailed methodology for
    running Apache Spark and Giraph using TeraHeap. The methodology
    includes conducting a research about the memory needs, i.e. Java
    Heap and IO Cache, of different big data analytics workloads when
    using Spark-Giraph with native JVM or JVM with TeraHeap.
    Understanding these needs helps us during the evaluation of
    running colocated instances.

    Our experimental results show that reducing Java Garbage
    Collection overhead by offloading the heap to fast storage devices
    significantly improves server throughput by up to 66\% against
    native Spark and Giraph. Finally, we also include a cost
    estimation to show that TeraHeap could reduce monetary cost by up
    to 50\% for running big data analytics, if deployed in a world
    cluster like Amazon's EC2 or Google Cloud Platform or Microsoft
    Azure Cloud, which are available to everyone.
\end{abstract}
