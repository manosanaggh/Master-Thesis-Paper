\begin{abstract}

    Managed big data analytic frameworks require a lot of memory to process
    large amounts of data.
    %
    The memory pressure that arises during the data processing
    prevents the frameworks from utilizing the CPU thus leading to
    low server throughput.
    %
	%    This pressure leads to long Garbage Collection
    %(GC) cycles
    %leaving no room for useful work. 
    In this paper, we conduct an analysis of server
    throughput for managed big data analytics. We evaluate a smart technique for heap
    offloading to fast storage devices in order to reduce memory pressure.
    %
    We use a framework called TeraHeap that moves objects from the Java
    managed heap to a secondary heap over a fast storage device. %By moving the data, TeraHeap frees
    %up heap memory and reduces memory pressure without suffering from storage latencies.  
    %
    %By overcoming the memory bound TeraHeap leaves space
    %to the applications for more CPU utilization.
    Our primary focus is to examine its performance under the colocation of
    multiple instances. 
    %
    Furthermore, we present a detailed methodology for running Apache Spark and Giraph using TeraHeap. The methodology
	includes conducting a research about the memory needs of Big data analytics when using offheap mechanisms
	i.e. Java Heap and IO Cache. 
	Understanding these needs helps us run the experiments with colocated instances.
	%reaking down the execution time and explaining
    %every different metric, provinding results such as server throughput...
	%TeraHeap is implemented in Oracle's OpenJDK8. In this paper we
    %evaluate its performance against Native Spark and Giraph using various
    %workloads of the Spark Bench suite and graphalytics library on a real-world cluster.
    %
    Our experimental results show that reducing memory pressure
    by offloading the heap to fast storage devices
    significantly improves server throughput and CPU utilization by reducing
    memory usage against native Spark and Giraph. We also include
    results to show that TeraHeap can reduce monetary cost for running Big data analytics if
    deployed in a world cluster like Amazon's EC2 or Google Cloud
    Platform or Microsoft Azure Cloud which are available to everyone. 
\end{abstract}
