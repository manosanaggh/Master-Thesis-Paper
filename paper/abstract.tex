\begin{abstract}
    Today, managed big data frameworks, such as Spark and Giraph
    demand large amount of memory to process massive volume
    datasets. The memory pressure that arises from the big data processing leads to
    long garbage collection (GC). In order to avoid this overhead, big
    data analytics frameworks, increase heap capacity and finally reach
    the memory limit of available DRAM in the server. Another approach
    is moving part of the managed heap to a storage device. However, this results to
    excessive Serialization/Deserialization (S/D) overheads. These limits
    prevent the frameworks from effectively utilizing the CPU thus
    leading to low server throughput.

    In this paper, we conduct a methodological analysis of server throughput for
    managed big data analytics frameworks. More specifically, we examine, whether
    storing data to fast storage devices instead of the main memory
    can help increase the effective CPU utilization of the server by reducing memory
    pressure. We use a system called TeraHeap that moves objects from
    the Java managed heap to a secondary heap over a fast storage
    device to reduce the GC overhead and eliminate S/D over data. We examine
    whether reducing these overheads leads to
    more effective utilization of the available CPU by the
    application. We focus on analyzing the system's
    performance under the collocation of multiple memory-bound
    instances. Our detailed methodology includes conducting a research about the memory needs, i.e. Java
    Heap and IO Cache, of different big data analytics workloads when
    using Spark and Giraph with native JVM or JVM with TeraHeap.
    Understanding these needs helps us during the evaluation of instance collocation.

    Our experimental results show that reducing Java Garbage
    Collection overhead by offloading the heap to fast storage devices
    significantly improves server throughput by up to 66\% against
    native Spark and... Finally, we also include a cost
    estimation to show that using an approach like TeraHeap could reduce monetary cost by up
    to 50\% for running big data analytics in a world
    cluster like Amazon's EC2 or Google Cloud Platform or Microsoft
    Azure Cloud, which are available to everyone.
\end{abstract}
