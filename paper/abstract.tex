\begin{abstract}
Managed big data frameworks, such as Apache Spark and Giraph
    demand a large amount of memory per core to process massive volume
    datasets effectively. The memory pressure that arises from the big data processing leads to
    high garbage collection (GC) overhead. Big data analytics frameworks attempt to remove this overhead by offloading objects to storage devices.
    At the same time, infrastructure providers, trying to address the same problem, attribute more memory to increase memory per instance leaving cores underutilized.
        For frameworks, trying to avoid GC through offloading to storage devices leads to high Serialiation/Deserialization (S/D) overhead.
        For infrastructure, the result is that resource usage is decreased.
    These limitations prevent managed big data frameworks from effectively utilizing the CPU thus leading to low server throughput.

    In this thesis, we conduct a methodological analysis of server throughput for
    managed big data analytics frameworks. More specifically, we examine, whether
    reducing GC and S/D can help increase the effective CPU utilization of the server.
    We use a system called TeraHeap (TH) that moves objects from
        the Java managed heap (H1) to a secondary heap over a fast storage
        device (H2) to reduce the GC overhead and eliminate S/D over data. We focus on analyzing the system's
    performance under the co-location of multiple memory-bound
    instances to utilize all available DRAM and study server throughput. Our detailed methodology includes choosing the DRAM budget for each instance and how to distribute
        this budget among H1 and Page Cache (PC). We try two different distributions for the DRAM budget, one with more H1
        and one with more PC to study the needs of both approaches.
    We evaluate both techniques under 3 different memory-per-core scenarios using Spark and Giraph with native JVM or JVM with TeraHeap.
	We do this to check throughput changes when memory capacity increases.

    Our experimental results show that increasing memory per core does not help reach max server throughput for analytics. Effective solutions
    for this problem is using systems like TeraHeap that offload objects from the managed heap without
    increasing the CPU load. Moving large parts of the heap to fast storage, decreases the DRAM GB per core needs and increases the utilization of the server. Finally, we also include a cost
    estimation to show that using an approach like TeraHeap could reduce monetary cost by up
    to 50\% for running big data analytics in a world
    cluster like Amazon's EC2 or Google Cloud Platform or Microsoft
    Azure Cloud, which are available to everyone.
\end{abstract}
